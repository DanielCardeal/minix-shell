\documentclass[12pt, twoside, a4paper]{article}

% --- Formatação doc ---
\usepackage[top = 2.5cm, bottom = 2cm, left = 2.5cm, right = 2.5cm]{geometry}
\usepackage{lmodern}  % fonte ABNT
\usepackage[T1]{fontenc}
\usepackage[portuguese]{babel}
\usepackage{gensymb}

% --- Utils ---
\usepackage{amsmath} % | matemática
\usepackage{amssymb} % |
\usepackage{graphicx}
\usepackage[shortlabels]{enumitem} % | configura enumerate

\usepackage{tabularx} % | Tabelas melhores
\usepackage{booktabs} % |

\usepackage{hyperref} % | Links clicáveis

\usepackage{xcolor} % | Texto colorido

% Anota TODOs no texto
\newcommand{\todo}[1]{{\color{red} TODO: #1}}

% Remove numero das seções
\makeatletter
\renewcommand{\@seccntformat}[1]{}
\makeatother

% Ajeita o espaçamento para equivalente ao 1.5
\linespread{1.5}

\title{
	Relatório EP1 --- Sistemas Operacionais
}

\author{
	Daniel Pessoa Cardeal, N\degree{}USP:10693170 \\
	Caio Danta Simão Ugêda, N\degree{}USP:11796868
}

\date{}

% --- Começo do documento ---
\begin{document}
\maketitle

\section{Explicação geral}%

A aplicação desenvolvida para este trabalho realiza a entrada de comandos e argumentos por meio de um \textit{loop} infinito e chamadas à função \verb|scanf|. Após a leitura, os comandos são mapeado a um \textbf{código de comando} (representado por um inteiro), o que simplifica a leitura do programa bem como previne a comparação extensiva de \emph{strings}. O código obtido é então passado para a função \verb|executa_comando|, que fica responsável por executar a rotina adequada com o argumento recebido.

\section{Item a}%

A lógica de mudança de permissão usando chamadas de sistema pode ser encontrada na função \verb|executa_chmod|. Essa recebe como parâmetros uma \emph{string} de formatação semelhante à usada em sistemas Linux (``755'', por exemplo), assim como o caminho (relativo ou absoluto) para o arquivo que deve ser modificado.

A \emph{string} de formatação é apenas um facilitador, e deve ser traduzida para um valor \textit{long} antes de ser passada para a chamada de sistema \verb|chmod|. Essa conversão é feita usando a função \verb|strtol| (\textit{string-to-long}) e a base octal, já que as permissões são dadas em triplas de valores entre 0 e 7.

Sabendo disso, a execução do comando ``protegepracaramba'' consiste simplesmente na chamada da função \verb|executa_chmod| com a permissão ``000'' (ou seja, nenhuma operação permitida) e o caminho recebido pelo usuário.

\section{Item b}%

O comando ``liberageral'' é idêntico em funcionalidade ao comando visto no item anterior, e só difere quanto à \textit{string} de permissão passada a função \verb|executa_chmod|. Neste segundo caso, passamos a \emph{string} ``777'', ou seja, todas as operações são permitidas para todos os usuários.

\section{Item c}%
\todo{}

\section{Item d}%
\todo{}

\end{document}
